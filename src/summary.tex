\documentclass[11pt,a4paper]{article}
\usepackage{CJKutf8}
\usepackage{indentfirst}
\usepackage[CJKbookmarks=true]{hyperref} 
\setlength{\parindent}{2em}
\renewcommand{\baselinestretch}{1.0}
\textwidth=155mm
\textheight=230mm
\topmargin=0pt
\oddsidemargin=0pt
\evensidemargin=0pt
\linespread{1.3}
\begin{document}
\begin{CJK}{UTF8}{gbsn}                       % 详情参阅CJK文件包

\title{服务学作业全集}
\author{刘佳亮}
\maketitle
\newpage
\renewcommand{\contentsname}{目录}
\tableofcontents
\newpage

\section{现代服务学案例}

\subsection{现代服务业案例之一 —— DigitalOcean}

DigitalOcean是一家美国的虚拟主机提供商,总部位于美国纽约。

DigitalOcean 由 Ben Uretsky 和 Moisey Uretsky 于2011年联合创立。2012年 DigitalOcean 成为了 TechStars 的孵化项目之一。不久之后, DigitalOcean 获得了来自 IA Ventures 领投的320万美元的首轮融资。2013年12月,Netcraft 放出的一份报告指出, DigitalOcean已经成为世界上发展速度最快的云主机服务提供商,在 Web 主机数量上超越了 Amazon Web Services 。2014年10月, DigitalOcean 超越 Rackspace 成为世界第四大主机服务提供商,拥有超过一百人的雇员和超过十六万的用户,每天新增加约一千个用户。

DigitalOcean提供的主要是面向开发者的云主机服务。云主机服务是建立在VPS技术之上的。所谓VPS(Virtual Private Server 虚拟专用服务器)技术,是指将一部服务器分割成多个虚拟专享服务器的服务。实现VPS的技术分为容器技术和虚拟化技术 。在容器或虚拟机中,每个VPS都可分配独立公网IP地址、独立操作系统、实现不同VPS间磁盘空间、内存、CPU资源、进程和系统配置的隔离,为用户和应用程序模拟出“独占”使用计算资源的体验。VPS可以像独立服务器一样,重装操作系统,安装程序,单独重启服务器。VPS为使用者提供了管理配置的自由,可用于企业虚拟化,也可以用于IDC资源租用。

云主机就是在VPS技术之上加上一个云的概念。云主机是在一组集群主机上虚拟出多个类似独立主机的部分,集群中每个主机上都有云主机的一个镜像,从而大大提高了虚拟主机的安全稳定性,除非所有的集群内主机全部出现问题,云主机才会无法访问。云主机是新一代的主机租用服务,它整合了高性能服务器与优质网络带宽,有效解决了传统主机租用价格偏高、服务品质参差不齐等缺点,可全面满足中小企业、个人站长用户对主机租用服务低成本,高可靠,易管理的需求。

DigitalOcean 提供的云主机服务拥有以下特点:
\begin{enumerate}
\item 快速部署,所有服务器能在55秒之内进行部署
\item 全部使用SSD硬盘,可以大幅度提高服务器的 I/O 性能
\item Tier-1带宽,所有服务器均有1Gb/s的带宽
\item 基于 KVM  虚拟,拥有相当高的安全性和稳定性;5.简单控制面板,易用而且方便
\item 强大的硬件,所有真机服务器使用 ECC 内存以及组成 RAID 的SSD存储阵列。

\end{enumerate}
\subsection{现代服务业案例之二 —— Namecheap}

Namecheap 是一家 ICANN 认证的域名注册和网站托管公司,总部位于美国加利福利亚州洛杉矶。

Namecheap 由 Richard Kirkendall 于2000年创立。今天有超过50域名客户和超过100万网站客户选择Namacheap。Namacheap声称其管理着超过三百万个域名。2012年,在 Lifehacker 举行的投票中, Namecheap 获得了“最受欢迎的域名注册机构”的称号。

域名(Domain Name),是由一串用点分隔的名字组成的Internet上某一台计算机或计算机组的名称,用于在数据在新的经济环境下,域名所具有的商业意义已远远大于其技术意义,而成为企业在新的科学技术条件下参与国际市场竞争的重要手段 ,它不仅代表了企业在网络上的独有的位置 ,也是企业的产品、服务范围、形象、商誉等的综合体现,是企业无形资产的一部分。同时,域名也是一种智力成果,它是有文字含义的商业性标记,与商标、商号类似,体现了相当的创造性。在域名的构思选择过程中,需要一定的创造性劳动,使得代表自己公司的域名简洁并具有吸引力,以便使公众熟知并对其访问,从而达到扩大企业知名度、促进经营发展的目的。可以说,域名不是简单的标识性符号,而是企业商誉的凝结和知名度的表彰,域名的使用对企业来说具有丰富的内涵,远非简单的“标识”二字可以穷尽。

域名可以简单的分成两类,一是国际域名(international top-level domain-names,简称iTDs),也叫国际顶级域名。这也是使用最早也最广泛的域名。例如表示工商企业的 .com,表示网络提供商的.net,表示非盈利组织的.org等;二是国内域名,又称为国内顶级域名(national top-level domainnames,简称nTLDs),即按照国家的不同分配不同后缀,这些域名即为该国的国内顶级域名。200多个国家和地区都按照ISO3166国家代码分配了顶级域名,例如中国是cn,美国是us,日本是jp等。在实际使用和功能上,国际域名与国内域名没有任何区别,都是互联网上的具有唯一性的标识。只是在最终管理机构上,国际域名由美国商业部授权的互联网名称与数字地址分配机构(The Internet Corporation for Assigned Names and Numbers)即 ICANN 负责注册和管理;而国内域名则由中国互联网络管理中心(China Internet Network Information Center) 即 CNNIC 负责注册和管理。Namecheap 就是 ICANN 认证的一家下属域名注册机构。

Namecheap 相对于其他域名注册机构,主要具有以下优点:
\begin{enumerate}
\item 提供免费DNS解析,网址转发(可隐藏原URL,支持301重定向)、电邮转发、A记录、CNAME别名记录、MX邮件记录、TXT文本记录、NS记录、AAAA记录(IPV6),还可以设置动态域名解析
\item 赠送whois隐藏和SSL证书免费一年,这是全球唯一一家域名注册商能够做到的
\item 支持在线PUSH,转移码在线获取,在付款上支持Paypal没有绑定信用卡的用户,这一点也是国外其他注册商无法做到的
\item 员工在域名管理,网站托管等领域有几十年的经验。
\end{enumerate}

\section{对服务系统、服务要素、服务模型、服务建模方法的认识}

\subsection{服务系统}
 
服务系统(Service System)可看作一种社会化的技术系统,是自然系统与制造系统的复合,是基于系统论基本观点而构造的一种框架,用来描述与解释一个组织内包含的技术与非技术要素直接的关联关系。服务系统是对特定的技术或组织的一种网络化配置,用来提供服务以满足顾客的需求和期望。在服务系统中,服务的提供者与服务的需求者之间按照特定的协议、通过交互以满足某一特定顾客的请求,进而创造价值,彼此之间形成协作生成关系。好的服务系统使得那些没有经验的服务提供者能够快速准确的完成复杂的服务任务。

服务系统是一种复杂系统,不仅是其内部各构成元素的“总和”,还包括它们之间的复杂交互关系。系统内部元素间的复杂交互是随机的、非线性的,行为难以预测。系统结构与配置可能随需求与资源提供的变化而频繁发生演化。同时,服务系统的重要要素是人,人的行为很难建模与仿真,也导致服务系统是动态的和开放的,形成一个复杂的自适应性系统。而且,服务系统控制权分散在各要素中,形成分布式控制系统。当发生变化时,系统各要素共同进行演化。 

服务系统的一个案例是医疗系统。在现代医疗系统中,医生,病人,其他人和与之相关的技术共同参与到一起,为了同一个基本目的——改善病人的身体健康。这些要素机器之间的交互共同构成了一个医疗“服务系统”。

\subsection{服务要素}

服务系统主要有以下要素组成:服务参与者,包括参与方及其组织,人力资源及其能力;服务资源,包括软件,硬件,资源能力,服务环境等;服务信息;服务交互行为等。

服务参与者覆盖了服务中“人”要素,又称为“服务主体”,包括顾客、提供者等服务参与方;服务参与者具有特定的价值需求,掌握特定的服务资源,可主动发出服务行为,通过彼此之间信息的共享完成服务。按服务参与者的层次不同,可将其分类为:组织,角色,人员及其能力。服务系统的参与者按照其角色可分为三类:顾客,提供者和使能者。

服务资源刻画了被动参与服务的各类支持性资源,具备特定的能力,在服务主体的控制下可向外提供特定的行为。服务资源主要包括环境、软件、硬件或设备等类型。“环境”是服务执行的物理类或IT类场地,软件、硬件、设备则是协助和支持人进行服务的支持设施。

服务信息刻画了服务交互行为过程中被创造或采集的各类信息(Shared Information),并在服务参与者之间、资源与参与者之间相互传递和使用。服务信息分为两类:资源类信息和指令类信息。资源类信息作为一种服务资源向服务行为提供支持(例如海运物流服务中的航线信息、舱位信息、报价信息等);指令类信息则是负责在参与者及其行为之间进行控制指令和业务信息的传递,以单据形式为主,决定着后续服务流程该如何运转 (例如,海运物流服务中的订舱单、海运十联单等)。

服务交互行为刻画服务系统内包含的各服务任务、过程、活动、动作,以及这些行为单元之间的交互关系。将一个服务项目(service project)分解为若干相对独立的服务任务(service task),每个服务任务可看作是供需双方参与者的一次协同生产过程(process)。服务过程由一组服务活动(activity)及其之间的时序关系构成,每个活动由特定参与方发出。服务活动可被继续分解为原子的服务动作(action)。

以大学教育系统为例,大学中的学生是顾客,教师和教学管理人员是服务提供者,教材,实验器材,校园网等是资源,专业知识,课程信息等是共享资源,教室,实验室等是环境,诸如教师提供课程,学生参与考核等是服务交互行为。

\subsection{服务模型与服务建模方法}

服务模型用于形式化描述和表示服务系统及各类服务要素(包括资源、能力、人员、行为、过程等)。服务建模方法的作用是服务系统的需求分析、设计和实施模型的转换与映射。

服务模型的主要作用有:1,表示服务与模型映射,通过模型来表达服务设计者的创新思想、业务需求、目标服务系统的形态(体系结构、构成要素),刻画系统顾客/使用者与服务要素之间的关联关系,进而建立这些模型之间的(半)自动映射与转换,以达到从需求获取/定义到系统最终实现的转化;2.沟通交流与协同设计,通过一套标准的建模符号,对各类服务要素(过程、资源、角色等)进行描述,从而在参与服务设计、开发、实施过程的各参与者之间建立一个交流与协作的平台。

服务建模是使用恰当的服务建模语言,按照特定的方法和步骤,构造出目标服务和服务系统的模型。服务建模是一个按照模型驱动方式将现实业务和需求转变成服务模型并最终指导系统实施生成的过程;通常不同的服务模型规范具有自身独特的建模方法。 

例如SMDA服务模型采用矩阵模型来刻画服务系统的静态结构与动态行为:横向的三个层次为服务需求获取->服务行为与能力规划->服务执行规划; 纵向的四条主线为刻画组织、行为、资源、信息等侧面。SMDA可以使用UML活动的一种多维图形表达方式——泳道图扩展成为双向泳道图来描述交互行为,其中纵向泳道图是是任务的执行者分割的,横向泳道是以任务分割的。


\section{服务构件概念以及服务构件类型}

\subsection{服务构件的概念}
 
服务系统内部包含的服务要素种类繁多。对服务提供者来说,这些服务要素应是可以被复用的,以构造面向不同需求的服务系统。服务要素进行复用的前提是,必须对服务要素做出全面描述并进行封装,以便于进行选取与组合。**服务构件(Service Component)**便是服务要素的一种统一封装。

对于服务构件的理解可以参考软件构件。软件构件是面向软件体系架构的可复用软件模块,是可复用的软件组成成份,可被用来构造其他软件。它可以是被封装的对象类、类树、一些功能、软件工程中的构件模块、软件框架、软件构架、文档、分析件、设计模式等。与之类似,服务构件是可被重复使用的、用来构造服务系统的基本单元,具有一系列特定的接口和描述。


\subsection{服务构件的类型}

常用的服务构件类型有如下几类:

\begin{enumerate}
\item Web Service 

    Web service是一个平台独立的,低耦合的,自包含的、基于可编程的web的应用程序,可使用开放的XML(标准通用标记语言下的一个子集)标准来描述、发布、发现、协调和配置这些应用程序,用于开发分布式的互操作的应用程序。
    
    Web Service技术, 能使得运行在不同机器上的不同应用无须借助附加的、专门的第三方软件或硬件,就可相互交换数据或集成。依据Web Service规范实施的应用之间,无论它们所使用的语言、平台或内部协议是什么,都可以相互交换数据。Web Service是自描述、自包含的可用网络模块,可以执行具体的业务功能。Web Service也很容易部署,因为它们基于一些常规的产业标准以及已有的一些技术,诸如标准通用标记语言下的子集XML、HTTP。Web Service减少了应用接口的花费。Web Service为整个企业甚至多个组织之间的业务流程的集成提供了一个通用机制。

\item WS-HumanTask

    WS-HumanTask是在WS基础上扩展,支持异步执行的人工活动 (对人工活动的虚拟化)的一种服务构件类型。
    人工任务(Human Task)是由人来”实现“的服务。人工任务有两个接口,一个接口暴露任务提供的服务,例如翻译服务,另一个接口允许人们完成这些任务。每一个服务都由人来负责,这些任务说明了谁应该在任务中担任某个角色。人工任务也可能指明任务的元数据应该如何在不同设备上处理。人工任务可以被定义为能够对超时做出反应,并执行恰当的补救操作。
    
\item REST (REpresentational State Transfer)

    REST(Representational State Transfer)是一种轻量级的Web Service架构风格,其实现和操作明显比SOAP和XML-RPC更为简洁,可以完全通过HTTP协议实现,还可以利用缓存Cache来提高响应速度,性能、效率和易用性上都优于SOAP协议。

    REST架构遵循了CRUD原则,CRUD原则对于资源只需要四种行为:Create(创建)、Read(读取)、Update(更新)和Delete(删除)就可以完成对其操作和处理。这四个操作是一种原子操作,即一种无法再分的操作,通过它们可以构造复杂的操作过程,正如数学上四则运算是数字的最基本的运算一样。

    REST架构让人们真正理解我们的网络协议HTTP本来面貌,对资源的操作包括获取、创建、修改和删除资源的操作正好对应HTTP协议提供的GET、POST、PUT和DELETE方法,因此REST把HTTP对一个URL资源的操作限制在GET、POST、PUT和DELETE这四个之内。这种针对网络应用的设计和开发方式,可以降低开发的复杂性,提高系统的可伸缩性。
  
\item SCA (Service Component Architecture)

    SCA提出的了一套基于SOA去构建企业应用的编程模型,它的基础思想就将业务功能构造成一系列的服务,并且能够很好地将这些服务组合起来,达到解决业务需求的目的。在构建这些应用时所用到的服务,不仅包含新建服务,而且可以包括已有的业务应用中的业务功能,也就是说, SCA 提供了一套针对服务组合和服务创建的模型。

    SCA是由IBM牵头,几家国内外知名企业联合制定的,于 2005 年 11 月发布了 0.9 版本,目前版本已经到了 0.96 。在 0.9 版本中。SCA 标准就提出了 Java 实现以及 C++ 实现标准,而且在以后的版本中,会陆续加入其他的实现标准,也就是说 SCA 并不是只针对某一种语言的,不同语言或者环境之间通过开放的,标准的技术来实现互操作,比如我们常见的WebService等。
    
\item SDO (Service Data Object)

    SDO是一种用于简化和统一应用程序处理数据的方式,编程人员可采用统一方式访问和操作来自异类数据源的数据,包括关系数据库、XML 数据源、Web 服务以及企业信息系统等。服务数据对象框架为数据应用程序开发提供了统一的框架。通过 SDO,开发人员不需要熟悉特定于技术的 API,就能访问和利用数据。开发人员只需要知道一种 API,即 SDO API,它允许开发人员处理来自多种数据源的数据,其中包括关系数据库、实体 EJB 组件、XML 页面、Web 服务、Java Connector Architecture、JavaServer Pages 页面等。
 
\item WS-Resource

    WS-Resource是指有状态的资源,这里的”资源“是指符合隐含资源模式所定义的交互模式的Web服务资源。因此,WS-Resource即表示有状态的资源,也表示与之相关联的Web服务。一个WS-Resource资源具有属性和状态,表达为XML形式,具有特定的生命周期,可被创建和销毁,通过若干特定的Web service对资源进行操作,改变其属性与状态。
\end{enumerate}


\section{面向服务的体系结构SOA简介}

\subsection{SOA含义}

面向服务的体系结构(Service Oriented Architecture)是一类分布式系统的体系结构,是实现由构件组成系统的模型。它将应用程序的不同功能单元(称为服务)通过服务之间定义好的接口和规范按松散耦合方式联系起来,通过网络整合成一个新的系统。其中接口是采用中立的方式进行定义的,独立于实现服务的硬件平台、操作系统和编程语言。这使得构建在各种系统中的服务可以以统一和通用的方式进行交互。


SOA是一种粗粒度、松耦合服务架构,服务之间通过简单、精确定义接口进行通讯,不涉及底层编程接口和通讯模型。SOA可以看作是B/S模型、XML(标准通用标记语言的子集)/Web Service技术之后的自然延伸。

SOA提升了将用户从服务实现分开的目标。服务可以运行在不同的服务器上,并通过网络被访问。 这也大大增加了服务的重用。SOA将能够帮助软件工程师们站在一个新的高度理解企业级架构中的各种组件的开发、部署形式,它将帮助企业系统架构者以更迅速、更可靠、更具重用性架构整个业务系统。较之以往,以SOA架构的系统能够更加从容地面对业务的急剧变化。

SOA的基本构件是“服务”。从外特性上看,一个服务被定义为显式的、独立于服务具体实现技术细节的接口。从内特性上看,服务封装了可复用的业务功能。它们通常是大粒度业务,如业务过程、业务活动等。服务实现可采用任何技术平台,如J2EE、.NET等。

\subsection{SOA技术架构}

SOA是一种架构模式/风格,并不单纯指一种架构。常见的SOA基本体系结构模式有如下几种:

\begin{enumerate}

\item 发布-访问模式

    发布-访问模式是由服务提供者在服务注册中心注册并发布服务描述,使服务请求者可以发现它,服务请求者通过查询服务注册中心找到满足其标准的服务,在检索到服务描述之后,服务使用者将根据服务描述中的信息来调用服务。

    著名的Web Service技术可以看成是发布-访问模式的一种实现,其中服务提供者使用WSDL描述服务接口,服务使用者使用UDDI发现相应服务并据此将服务集成在自身的系统中。服务提供者和服务使用者之间通过SOAP协议交换信息。

\item 适配器模式

	适配器模式是针对企业中存在若干遗存系统(Legacy System),遗存系统采用较传统的技术开发,无法提供清晰的接口,并且其他系统仍需要访问这些遗留系统的功能。适配器模式的解决途径类似设计模式中的适配器模式,通过构造适配器(Adaptor, Wrapper),将遗存系统中的功能进行二次封装,并开放出接口供其他系统使用;从而使非面向服务的系统也能够参与到面向服务的体系结构之中。支持适配器模式的技术有Java 2 Connector和IBM WebSphere Business Integration Adaptor等。
	
	
\item 远程服务策略

    远程服务策略是一种客户端直接绑定服务接口(WSDL/URI),客户端通过“Service Registry”来访问服务的模式,当希望访问其他服务时,只要手工修改Registry即可,客户端通过“Service Broker”来动态决定需访问那个服务。
	
	远程服务策略提供根据服务质量或者功能考虑事项改变服务提供者的灵活性。这使得在合并应用程序组合时加速合并和采购以及灵活地改变提供者成为可能。
	
\item 服务集成器

    服务集成器模式是在企业应用集成器EAI基础上发展起来的一种服务集成方式。
     EAI(Enterprise Application Integration),是企业应用集成。EAI是将基于各种不同平台、用不同方案建立的异构应用集成的一种方法和技术。EAI通过建立底层结构,来联系横贯整个企业的异构系统、应用、数据源等,完成在企业内部的 ERP、CRM、SCM、数据库、数据仓库,以及其他重要的内部系统之间无缝地共享和交换数据的需要。有了 EAI,企业就可以将企业核心应用和新的Internet解决方案结合在一起。
     
\item 企业服务总线(ESB)

    ESB全称为Enterprise Service Bus,即企业服务总线。它是传统中间件技术与XML、Web服务等技术结合的产物。ESB提供了网络中最基本的连接中枢,是构筑企业神经系统的必要元素。ESB在请求者和服务之间实现了路由服务间的消息,转化请求者和服务之间的传输协议,转换请求者和服务之间的消息格式,处理分离资源间的业务事件。
    
    ESB的出现改变了传统的软件架构,可以提供比传统中间件产品更为廉价的解决方案,同时它还可以消除不同应用之间的技术差异,让不同的应用服务器协调运作,实现了不同服务之间的通信与整合。从功能上看,ESB提供了事件驱动和文档导向的处理模式,以及分布式的运行管理机制,它支持基于内容的路由和过滤,具备了复杂数据的传输能力,并可以提供一系列的标准接口。
    
\end{enumerate}

\section{大服务与务联网}
\subsection{下一代互联网——务联网}

欧盟在第七框架的“未来互联网”计划中提出了务联网(Internet of Service)概念。这是继计算机联网、终端联网、实物联网基础上进一步建立起的各类服务联网的新兴应用方向。所有需要使用软件应用的事务或事物都可以以互联网上的服务形式存在,如软件、软件开发工具、软件运行平台等。务联网随同人际网(社会化网络)、知识/内容网、物联网一道,共同构成了未来互联网和网络化社会的四大支柱。

务联网是继互联网、物联网之后信息技术发展的一种新形式。互联网实现了虚拟数字空间,支持信息的互联互通和信息资源共享;物联网实现了物理/数字空间,支持物理世界与信息世界的互联互通与资源共享的融合;务联网则要实现现实/虚拟应用空间,支持现实世界与信息世界的资源共享与应用服务的融合。

务联网依托互联网实现的现实世界与数字世界的网络化应用服务形态和聚生态系统,以集成服务的形式支持网络环境下的各种现实服务的实现,如生产性商务服务、生活消费服务、社会服务、信息服务等。在务联网中,所有的服务、人、资源、物体均可通过网络访问之,各种服务均可结合网络提供之。


务联网在云计算环境下,除了体现服务资源及服务系统的网络泛在化与虚拟化,还更多地强调以服务的形式支持软件服务与商务服务。务联网支持各类服务参与者(顾客、提供者、使能者)间、多服务间进行基于网络的服务协作与交易。各种服务被配置于网络中,在网络环境下被发布、标注,被使用者或使能者所发现、聚合、集成,在专门的服务使能者支持下通过多种业务和技术渠道进行服务交付,并获取顾客的满意及参与者的价值。

\subsection{大数据时代}

大数据技术(Big Data),或称巨量资料,指的是所涉及的资料量规模巨大到无法通过目前主流软件工具,在合理时间内达到撷取、管理、处理、并整理成为帮助企业经营决策更积极目的的资讯。在维克托·迈尔-舍恩伯格及肯尼斯·库克耶编写的《大数据时代》中大数据指不用随机分析法(抽样调查)这样的捷径,而采用所有数据进行分析处理。

大数据”这个术语最早期的引用可追溯到Apache的开源项目Nutch。当时,大数据用来描述为更新网络搜索索引需要同时进行批量处理或分析的大量数据集。随着谷歌MapReduce和Google File System (GFS)的发布,大数据不再仅用来描述大量的数据,还涵盖了处理数据的速度。

“大数据”是需要新处理模式才能具有更强的决策力、洞察发现力和流程优化能力的海量、高增长率和多样化的信息资产。大数据具有如下4V特点:Volume(海量数据规模,数据量在TB级到PB级)、Velocity(快速处理,快速数据流转和动态数据体系)、Variety(多样数据类型,数据类型繁杂)、Value(数据价值,价值稀疏,多样而不确定)。

大数据技术的战略意义不在于掌握庞大的数据信息,而在于对这些含有意义的数据进行专业化处理。换言之,如果把大数据比作一种产业,那么这种产业实现盈利的关键,在于提高对数据的“加工能力”,通过“加工”实现数据的“增值”。

\subsection{从大数据到大服务}

大服务是由跨世界(现实世界与数字世界)、跨领域、跨区域、跨网络的服务经过聚合与协同而形成的服务生态系统。大服务基于大数据蕴含规律,产生各种智能业务服务,构成跨网跨域跨世界的复杂服务生态系统或务联网,用于解决企业或社会中大数据关联业务处理与业务应用问题。

大服务的基本特征可以简单概括为MC4V,即Massive(大),Complex(杂),Cross(跨), Convergent(聚),Customized(客),Value(价值)。具体的特征如下:

\begin{itemize}

\item 大规模服务:复杂服务、规模大、类型杂、来源多、无处不在
\item 服务聚合性:大数据环境下应用业务服务由来自不同服务系统或跨网跨域的服务经多次聚合动态构成
\item  顾客需求大规模个性化:服务系统面向大规模顾客个性化需求、个性化定制服务
\item 服务价值性:追求服务各方的价值最大化和价值知觉过程
\item 面向顾客的服务质量:以顾客满意度为目的的服务质量保障
\item 智能服务提供:面向用户体验与满意度的智能主动服务提供
\item 服务可信性:开放环境下复杂服务系统要具有更高的可信性、安全性与可靠性

\end{itemize}

\clearpage\end{CJK}
\end{document}
